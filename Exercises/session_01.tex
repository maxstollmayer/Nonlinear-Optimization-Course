\documentclass[11pt, a4paper]{article}

\usepackage[utf8]{inputenc}
\usepackage[T1]{fontenc}
\usepackage{lmodern}
\usepackage[english]{babel}
\usepackage{amsmath}
\usepackage{amsfonts}
\usepackage{amssymb}
\usepackage{dsfont}
\usepackage{parskip}

\newcommand{\N}{\mathbb{N}}
\newcommand{\Z}{\mathbb{Z}}
\newcommand{\R}{\mathbb{R}}
\renewcommand{\C}{\mathbb{C}}
\renewcommand{\epsilon}{\varepsilon}
\renewcommand{\phi}{\varphi}


\begin{document}

# Exercise Session 1

## Problem 1

Let $X \subseteq \mathbb{R}^n$ be a nonempty convex set and $f : \mathbb{R}^n \to \mathbb{R}$ a convex function. Show that every local minimum of $f$ with respect to $X$ is a global minimum of $f$ with respect to $X$.

### Solution

Let $y^*$ be a global minimum and $x^*$ be a local minimum. If $f(x^*) = f(y^*)$ we are be done, so suppose that $f(y^*) < f(x^*)$.

Since $f$ is convex we have for all $\lambda \in [0, 1)$

\begin{align*}
f \big( \lambda x^* + (1-\lambda) y^* \big) &\le \lambda f(x^*) + (1-\lambda) f(y^*) \\
&< \lambda f(x^*) + (1-\lambda) f(x^*) \\
&= f(x^*)
\end{align*}

Since $X$ is convex there exists a $\lambda \in [0,1)$ such that $\lambda x^* + (1-\lambda) y^*$ is in a neighbourhood in which $x^*$ is the local minimum.

But then $f(x^*) \le f \big( \lambda x^* + (1-\lambda) y^* \big) < f(x^*)$, which is a contradiction. So it must be that $f(x^*) = f(y^*)$ and thus $x^*$ is also a global minimum.



\end{document}